\chapter{Introduzione}
\section{Argomenti del corso}
\begin{itemize}
  \item Approssimazione di dati e funzioni:
  \begin{itemize}
    \item interpolazione polinomiale
    \item matrice di vandermonde
    \item interpolazione di Lagrange
    \item interpolazione di Hermite
    \item definizione di differenza divisa
    \item interpolazione (alla Newton) alle differenze divise
    \item convergenza
    \item controesempio di Runge su nodi equispaziati
    \item rappresentazione dell'errore
    \item funzioni a tratti splines
    \item interpolazione con funzioni splines
    \item metodo dei minimi quadrati 
    \item Cenno curve di Bézier
    \item cenno interpolazione in più dimensioni
  \end{itemize}
  \item Integrazione numerica
  \begin{itemize}
    \item formula quadratica di interpolazione
    \item formule di Newton-Cotes
    \item studio dell'errore e delal convergenza
    \item routines automatiche
    \item uso di formule per integrali in più dimensioni
  \end{itemize}
  \item Sistemi lineari
  \begin{itemize}
    \item motodi diretti
    \item sistemi a matrice triangolare
    \item metodo di eliminazione di Gauss
    \item pivoting
    \item decomposizione di Gauss e fattorizzazione a LU
    \item metrice inversa
    \item raffinamento iterativo
    \item sistemi complessi
    \item Cenni a metodi iterativi di jacobi e di Gauss-Seidel
    \item studio della convergenza dei metodi iterativi e criteri di arresto
  \end{itemize}
  \item Equazioni non lineari
  \begin{itemize}
    \item radici reali
    \item metodo di Newton-Raphson
  \end{itemize}
  \item Matlab
  \item Prerequisiti
  \begin{itemize}
    \item operazioni tra matrici
    \item matrici non singolari
    \item determinante
    \item cramer
    \item regola di Laplace
    \item matrice inversa
    \item concetto ddi norma
    \item norma di un vettore e di una matrice
    \item lineare dioendenza e indipendenza
  \end{itemize}
\end{itemize}
\section{Modalità d'esame}
Prova scritta su esercizi e prova orale di 1 ora :(
