\chapter{Interpolazione}

\section{Lezione 1}

L'interpolazione permette di semplificare funzioni complesse in polinomi che si lasciano studiare.

Vale il seguente teorema:

\textbf{Teorema}: Se $(x_i, y_i)$ con $i = 0, \dots, n$, sono $n + 1$ punti tale che $x_i \neq x_j$
se $i \neq j$, allora esiste ed è unico il polinomio $p_n(x)$ di grado al più $n$ tale che:


\begin{equation}
    p_n(x_i) = y_i, \quad \forall i \in 0, \dots, n
\end{equation}


\textbf{Dimostrazione}: Considero il generico polinomio di grado $n$:
\begin{equation}
    p_n(x) = a_0 + a_1x + a_nx^n
\end{equation}

ed impongo le $n + 1$ condizioni(o vincoli)

\begin{equation}
    \label{eq:1.3}
    \begin{cases}
        a_0 + a_1x_0 + a_nx_0^n = y_0 \\
        a_0 + a_1x_1 + a_nx_1^n = y_1 \\
        \dots \\
        a_0 + a_nx_0 + a_nx_n^n = y_n 
        
    \end{cases}
\end{equation}

I parametri(o coefficienti) incogniti $a_0, a_1, \dots, a_n$ sono soluzioni del sistema lineare (\ref{eq:1.3})
di ordine $n + 1$.

Ora trasporto tutto nella forma matriciale e ottengo:


\begin{equation}
    \left(
        \begin{array}{c c c c}
            1 & x_0 & \dots & x_0^n \\
            1 & x_1 & \dots & x_1^n \\
            \dots & \dots & \dots & \dots \\
            1 & x_n & \dots & x_n^n
        \end{array}
    \right)
    \left(
        \begin{array}{c}
            a_0 \\
            a_1 \\
            \dots \\
            a_n
        \end{array}
    \right)
    =
    \left(
        \begin{array}{c}
            y_0 \\
            y_1 \\
            \dots \\
            y_n
        \end{array}
    \right)
\end{equation}

che si può anche scrivere così:
\begin{equation}
    V \cdot a = y
\end{equation}

dove $V$ è la matrice di \textbf{Vandermonde}, questa matrice risulta non singolare se e solo se
il vettore nullo è la sola soluzione del sistema omogeneo:
\begin{equation}
    V \cdot a = 0
\end{equation}

Così se il vettore delle $a$ è diverso da 0 possiamo costruire il polinomio di grado $n$.
\begin{equation}
    p_n(x) = a_0 + a_1x + \dots + a_nx^n
\end{equation}

Il vettore $a$ deve essere nullo di modo da avere la matrice $V$ non singolare.

In alternativa si può trovare il determinante della matrice e vedere che non sia nullo:
\begin{equation}
    det(V) = \prod_{i>j}{(x_i-x_j)}
\end{equation}

Se il determinante è diverso da zero, il sistema ammette una ed una sola soluzione, quindi
il polinomio $p_n(x)$ esiste ed è unico.


Provo ora ad usare una base diversa da quella canonica per rappresentare il polinomio,
prendo per esempio tre punti distinti $x_0, x_1, x_2$.

Mi trovo ora i nuovi polinomi $L$:
\begin{equation}
    L_0(x) = \frac{
        (x - x_1)(x - x_2)
    }{
        (x_0 - x_1)(x_0 - x_2)
    }
\end{equation}
\begin{equation}
    L_1(x) = \frac{
        (x - x_0)(x - x_2)
    }{
        (x_1 - x_0)(x_1 - x_2)
    }
\end{equation}
\begin{equation}
    L_2(x) = \frac{
        (x - x_0)(x - x_1)
    }{
        (x_2 - x_0)(x_2 - x_1)
    }
\end{equation}

Costrundo così le $L$, si ottiene una caratteristica carina:
\begin{equation}
    L_0(x_0) = 1, \quad L_0(x_1) = 0, \quad L_0(x_2) = 0
\end{equation}

\begin{equation}
    L_1(x_0) = 0, \quad L_1(x_1) = 1, \quad L_1(x_2) = 0
\end{equation}

\begin{equation}
    L_2(x_0) = 0, \quad L_2(x_1) = 0, \quad L_2(x_2) = 1
\end{equation}


Posso quindi esprimere un \textbf{generico polinomio di secondo grado} come combiazione dei tre
nuovi polinomi costruiti:
\begin{equation}
    p_2(x) = a_0L_0(x) + a_1L_1(x) + a_2L_2(x)
\end{equation}

Assegnando tre valori arbitrari di $y$ si ottiene una \textbf{nuova base lagrangiana}, ottenendo il sistema:
\begin{equation}
    \begin{cases}
        c_0 L_0(x_0) + c_1 L_1(x_0) + c_2 L_2(x_0) = y_0 \\
        c_0 L_0(x_1) + c_1 L_1(x_1) + c_2 L_2(x_1) = y_1 \\
        c_0 L_0(x_2) + c_1 L_1(x_2) + c_2 L_2(x_2) = y_2
    \end{cases}
\end{equation}

Il polinomio interpolatore viene scritto come:
\begin{equation}
    p_2(x) = y_0 L_0(x) + y_1 L_1(x) + y_2 L_2(x)
\end{equation}

\section{Lezione 2}

Per generalizzare la lezione precedente dal grado 2 al grado $n$, questa è la skills:


\begin{equation}
    p_n(x) = \sum_{j = 0}^{n}{y_j L_j(x)}, \quad \text{dove} \quad L_j(x) = \prod_{i = 0, i \neq j}^{n}{\frac{
        x - x_i
    }{
        x_j - x_i
    }}
\end{equation}


Posso riscrivere il polinomio come:
\begin{equation}
    p_n(x) = \sum_{j = 0}^{n}{y_j L_j(x)} = \omega_n(x) \sum_{j = 0}^{n}{\frac{
        \beta_j
    }{
        (x - x_j)
    }}
\end{equation}

Dove:
\begin{equation}
    \omega_n(x) = \prod_{j = 0}^{n}{(x - x_j)}, \quad \beta_j = \frac{
        y_j
    }{
        \prod_{i = 0, i \neq j}^{n}{(x_j - x_i)}
    }
\end{equation}


Le operazioni richieste per calcolare un punto diverso da $x$ sono:
\begin{enumerate}
    \item $n$ sottrazioni
    \item $\frac{n^2}{2}$ sottrazioni
    \item $n^2$ moltiplicazioni
    \item $n$ addizioni
    \item $2n$ moltiplicazioni
\end{enumerate}

Arrivato a pagina 4