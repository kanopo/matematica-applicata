\chapter{Interpolazione}
\section{Curve di Bézier}
Siano i punti $P_i(x_i, y_i)$ con $i = 0, \dots, n$, $n+1$ punti di un sistema di 
coordinate cartesiane per $R^2$ chiamati punti di controllo.

Genero una curva parametrica polinomiale che rappresenti ogni punto con:
\begin{equation}
  Q_n(t) = \sum_{i = 0}^{n} P_if_i(t) \quad t \in [0, 1]
\end{equation}


Dove $f$ sono funzioni polinomiali scelte in modo che la curva:
\begin{itemize}
  \item $Q_n(0) = P_0$
  \item $Q_n(1) = P_1$
  \item La tangente in $P_0$ sia parallela a $P_1 - P_0$(parallela al segmento che unisce 
  p1 e p0)
  \item Le funzioni $f_i(t)$ siano simmetriche rispetto a $t$ e $1 -t$
\end{itemize}


Queste funzioni sono prese per vere se usaimo il polinomio di Bernstein:
\begin{equation}
  B_i^n(t) = \binom{n}{i}t^i(1 - t)^{n-i}, \quad i = 0, \dots, n
  \label{eq:bernstein}
\end{equation}

Dove:
\begin{equation}
  \binom{n}{i} = \frac{n!}{i!(n-i)!}
\end{equation}

\subsection{Curve di Bazier razionali}
Dati $n+1$ punti di controllo $P_i$, la curva di Bazier razionale è data da:
\begin{equation}
  C_n(t) = \frac{\sum_{i=0}^{n}B_i^n(t)P_iw_i}{\sum_{i=0}^{n}B_i^n(t)w_i} = \frac{\sum_{i=0}^n\binom{n}{i}t^i(1-t)^{n-i}P_iw_i}{\sum_{i=0}^n\binom{n}{i}t^i(1-t)^{n-i}w_i}
\end{equation}

Dove $w_i$ sono sono dei pesi(costanti positive), il numeratore è una curva di bazier pesata
in forma di Bernstein, mentre il denominatore è una somma pesata di polinomi di Bernstein.

\section{Interpolazione di funzioni in più variabili}

Per fare l'interpolazione in più variabili si procede come se si dovesse fare l'interpolazione
in mono variabile, avendo una funzione $f(x, y)$ definita in un rettangolo:
\begin{equation}
  R := \{(x, y): a\leq x \leq b, c \leq y \leq d\}
\end{equation}

Creo le due decomposizioni $\Delta_x$ e $\Delta_y$ dove:
\begin{equation*}
  \Delta_x = \{a = x_0 < x_1 < \dots < x_n = b\}
\end{equation*}

\begin{equation*}
  \Delta_y = \{c = y_0 < y_1 < \dots < y_m = d\}
\end{equation*}

Posso costruire i due polinomi lagrangiani per $x$ e $y$ e poi "assemblarli" in:
\begin{equation}
  p_{n,m}(x_i, y_j) = f(x_i, y_j) = L_{i,j}(x, y) = L_i(x)L_j(y)
\end{equation}

Sapendo che i polinomi lagrangiani sono uguali a 0 quando:
\begin{equation*}
  L_{i,j}(x_k, y_l) = 0 \quad \text{per} \quad k \neq i \quad \text{o} \quad l \neq j
\end{equation*}

Il polinomio interpolatore diventa:
\begin{equation}
  p_{n,m} (x, y) = \sum_{i=0}^n\sum_{j=0}^m f(x_i, y_j) L_i(x)L_j(y)
\end{equation}

Questo metodo di scomporre le interpolazioni multiple in singole e riassemblarle alal fine 
si può applicare sia al polinomio di Newton che alle splines.
