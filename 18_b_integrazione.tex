\chapter{Integrazione}
\section{Quadratura di Cavalieri-Simpson}
Voglio approssiamre l'integrale $f$ continuo nell'intervallo $[a, b]$
\begin{equation*}
  I(f) = \int_a^b f(x)dx
\end{equation*}

\begin{enumerate}
  \item Decompongo l'intervallo in $n=2$ sotto intervalli di ampiezza $h=\frac{b-a}{2}$
  \item Approssimo la funzione integranda $f(x)$ con un polinomio $p_2(x)$ costruito interpolando $f(x)$ negli estremi dell'intervallo
\end{enumerate}

Le $x$ e le $f(x)$ degli intervalli risultano:
\begin{itemize}
  \item $(a, f(a))$
  \item $(\frac{a+b}{2}, f(\frac{a+b}{2}))$
  \item $(b, f(b))$
\end{itemize}

Dai quali posso costruire il polinomio:
\begin{equation*}
  p_2(x) = f(a)L_0(x) + f(\frac{a+b}{2})L_1(x) + f(b)L_2(x)
\end{equation*}

Dove:
\begin{equation*}
  L_0(x) = \frac{(x - \frac{a+b}{2})(x - b)}{(a - \frac{a+b}{2})(a - b)}
\end{equation*}

\begin{equation*}
  L_1(x) = \frac{(x - a)(x - b)}{(\frac{a+b}{2} - a)(\frac{a+b}{2} - b)}
\end{equation*}

\begin{equation*}
  L_2(x) = \frac{(x - a)(x - \frac{a+b}{2})}{(b - a)(b - \frac{a+b}{2})}
\end{equation*}

Per comodita' cambio $a, \frac{a+b}{2}$ e $b$ con $x_0, x_1, x_2$.

Riprendo l'integrale iniziale e scambio la funzione con quello ottenuto:
\begin{equation}
  \int_a^b f(x)dx \approx \int_{x_0}^{x_2} p_2(x)dx = \int_a^b \sum_{i=0}^2 f(x_i)L_i(x)dx = \sum_{i=0}^2 w_i^{(2)}f(x_i)
\end{equation}

Dove:
\begin{equation}
  w_i^{(2)} = \int_{x_0}^{x_2} L_i(x)dx
\end{equation}


I punti $x_i$ e le costanti $w_i^{(2)}$ sono i \textbf{nodi e pesi} della formula di quadratura interpolatoria.

I pesi sono:
\begin{itemize}
  \item $w_0^{(2)} = \frac{h}{3}$
  \item $w_1^{(2)} = \frac{4h}{3}$
  \item $w_2^{(2)} = \frac{h}{3}$
\end{itemize}

Le costanti applicate alle $h$ sono:
\begin{itemize}
  \item $\alpha_0 = 1/3$
  \item $\alpha_1 = 4/3$
  \item $\alpha_2 = 1/3$
\end{itemize}

Queste costanti vengono chiamate costanti di Newton-Cotes, non dipendono dalla funzione integranda e dall'intervallo di integrazione.

Da notare che sommando le $\alpha_0 + \alpha_1 + \alpha_2 = 2 = n$.
Si nota inoltre che $\alpha_0 = \alpha_2$ e che possiamo calcolare $\alpha_1 = 2 - (\alpha_0 + \alpha_2)$ e che quindi basta sapere solo una $\alpha$.
\subsection{Approssimazione integrale con formula di Cavalieri-Simpson}
La formula finale e':
\begin{equation}
  \int_a^b f(x)dx \approx w_0^{(2)}f(a) + w_1^{(2)}f(\frac{a+b}{2}) + w_2^{(2)}f(b) = \frac{h}{3}f(a) + \frac{4h}{3}(\frac{a+b}{2}) + \frac{h}{3}f(b)
\end{equation}

Andando avanti si ottiene:
\begin{equation}
  \int_a^b f(x)dx \approx \frac{b-a}{6}[f(a) + 4f(\frac{a+b}{2}) + f(b)]
\end{equation}
Essendo un'approssimazione e' chiaro che devo avere un resto da qualche parte...

\subsection{Resto di Cavalieri-Simpson}




Il resto e' dato da:
\begin{equation}
  \int_a^b [f(x) - p_2(x)]dx = \int_a^b [(x-a)(x-\frac{a+b}{2})(x-b)]\frac{f^{(3)}(\varepsilon_x)}{3!}dx
\end{equation}

Si puo' dimostrare che se la funzione da integrare $\in C_4([a, b])$ il resto della formula di Cavalieri-Simpson e':
\begin{equation}
  r(CS_N) = - \frac{(b-a)^5}{2880}f^{(4)}(\varepsilon) = - \frac{h^5}{90}f^{(4)}(\varepsilon)
\end{equation}

Dove la $h$ e':
\begin{equation}
   h = \frac{b-a}{2}
\end{equation}

\section{Formula di Cavalieri-Simpson composita}
Usando $N$ sotto intervalli di suddivisione di $[a, b]$ con $N$ pari ($N = 2, 2^2, \dots$)
l'errore decresce come $N^{-4}$.

Per esempio:
\begin{equation}
  \frac{|R_N|}{|R_{2N}|} \approx \frac{1: (N)^{4}}{1: (2N)^{4}} = 16
\end{equation}

Quindi raddoppiando gli intervalli il rapporto fra gli errori massimio e' 16.


\section{Considerazioni sull'integrazione}
\begin{itemize}
  \item La formula dei trapezi ha grado di precisione 1(integra esattamente tutti i polinomi di grado minore o uguale a 1)
  \item la formula di cavalieri-Simpson ha grado di precisione 3(integra esattamente tutti i polinomi di grado minore o uguale a 3)
\end{itemize}
