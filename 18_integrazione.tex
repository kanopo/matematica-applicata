\chapter{Integrazione}

\section{Esempi di problemi}
Il significato degli integrali definit dipende dal contesto applicativo di cui fanno
parte.

\subsection{Spazio percorso in moto rettilineo}
Lo spazio percorso con velocità costante $v$ in un tempo $t = b - a$ è dato da:
\begin{equation*}
  s = v(b-a)
\end{equation*}

Nel caso la velocità non fosse costante ma variasse nel tempo con una funzione:
\begin{equation*}
  v = f(t)
\end{equation*}

Bisognerebbe:
\begin{enumerate}
  \item dividere l'intervallo in $n$ parti uguali
  \item l'ampiezza dell'intervallo sarebbe $h = (b - a) / n$
  \item il tempo di ogni intervallo sarebbe $t_i = a + ih$ dove $i = 0, 1, \dots, n$
  \item si approssima la velocità dell'intervallo alla velocità iniziale dello stesso
  \item Lo spazio finale percorso dal mobile risulta $\sum_{i = 0}^{n-1} f(t_i)h$
\end{enumerate}

Si ottiene così:
\begin{equation*}
  s =\int_a^b f(t) dt
\end{equation*}
Questa è una lunghezza.

\subsection{Lavoro compiuto da una forza}
Avendo una forza con intensità variabile applicata ad un punto che si sposta posso ricavare come prima:

\begin{equation*}
  L =\int_a^b f(x) dx
\end{equation*}
Questo è un lavoro.
\subsection{Valore medio di una funzione}
\begin{equation*}
  \frac{1}{b-a}\int_a^b f(x)dx
\end{equation*}

\section{Quadratura interpolatoria dei trapezi}
\begin{equation}
  T_N(f) = \frac{b-a}{2N}[f(a)+ 2\sum_{i=1}^{N-1}f(x_i) + f(b)] 
\end{equation}

Si può fare quando una funzione reale è definita su un intervallo chiuso e limitato $[a, b]$.

Quando si approssima l'integrale $I(f)$ con i trapezi $T_N(f)$ l'errore è dato da:
\begin{equation}
  r(T_n) = I(f) - T_N(f) = \sum_{i=0}^{N-1}r_i
\end{equation}
Dove $r_i$ è il resto per l'iesimo intervallo.

\subsection{Teoremi integrazione}
\subsubsection{Teorema 1}
se $f: [a, b]  \Rightarrow R$ è continua(integrabile) allora esiste un punto $c \in [a, b]$ tale che:
\begin{equation*}
  \int_a^b f(x)dx = (b-a)f(c)
\end{equation*}

\subsubsection{Teorema 2}
Se nell'integrale $\int_a^b f(x)g(x)dx$ una delle due funzioni, suppongo $g(x)$, è di segno
costante su tutto l'intervallo allora esiste un punto $c \in [a, b]$ tale che:
\begin{equation*}
  \int_a^b f(x)g(x)dx = f(c)\int_a^b g(x)dx
\end{equation*}


\subsection{Resto}
Posso usare la teoria degli integrali di prima per poter riscrivere il resto come:
\begin{equation}
  r_i = -\frac{h^3}{12}f''(c_i)
\end{equation}

Dove $c$ è un punto opportuno nell'intervallo $x_i, x_i +1$, posso applicare questa relazione su tutti gli intervalli
e ottengo il resto totale della quadratura come:
\begin{equation}
  r(T_N) = \sum_{i = 0}^{N-1} r_i = -\frac{h^3}{12}\sum_{i = 0}^{N-1}  f''(c_i) = -\frac{h^3}{12} N f''_M
\end{equation}

Dove $f''_M$ è la media degli $N$ valori $f''(c_i)$.

Esiste un punto $\varepsilon \in (a, b)$ tale che $f''(\varepsilon) = f''_M$.



Si ottiene l'espressione del resto della formula dei trapezi:
\begin{equation}
  r(T_N) = -\frac{1}{12}(\frac{b-a}{N})^3 N f''(\varepsilon) = -\frac{(b-a)^3}{12N^2}f''(\varepsilon)
\end{equation}
Oppure posso scrivere:
\begin{equation}
  r(T_N) = -\frac{1}{12}\frac{(b-a)3}{N^2} f''(\varepsilon) = -\frac{h^2}{12}(b-a)f''(\varepsilon), \quad h = \frac{b-a}{N}
\end{equation}

\section{Alcuni richiami}
\subsection{Teorema 1}
Se ho due funzioni continue nell'intervallo $[a, b]$ allora:
\begin{equation*}
  \int_a^b [c_1f_1(x) + c_2f_2(x)]dx = c_1\int_a^b f_1(x)dx + c_2\int_a^b f_2(x)dx
\end{equation*}

\subsection{Teorema 2}
Se una funzione e' continua in $[a, b]$ allora:
\begin{equation*}
  |\int_a^b f(x)dx| \leq \int_a^b |f(x)|dx
\end{equation*}

\subsection{Corollario}
Se una funzione e' continua in $[a, b]$, si ha:
\begin{equation*}
  m(b-a) \leq \int_a^b f(x)dx \leq M(b-a)
\end{equation*}

Dove $m$ e' il valore piu' piccolo presente nell'intervallo $[a, b]$, mentre $M$ e' il massimo valore
dell'intervallo.

Inoltre esiste $c \in [a, b]$ tale che:
\begin{equation*}
  \int_a^b f(x)dx = f(c)(b-a)
\end{equation*}

Usando il teorema 1, si puo' scrivere $m \leq \mu \leq M$ dove :
\begin{equation*}
  \frac{1}{b-a}\int_a^b f(x)dx = \mu
\end{equation*}

Che significa che avendo la funzione $f$ continua nell'intervallo, assume tutti i valori compresi tra minimoi e massimo.

Per il secondo teorema della media integrale si sfrutta il teorema di Weierstrass,
Sia $[a,b] \subset R$ un intervallo chiuso e limitato non vuoto e sia $f$ uan funzione continua che va da 
$[a,b] \Rightarrow R$.

Allora $f(x)$ ammette almeno un punto di massimo assoluto e un punto di minimo assoluto nell'intervallo
$[a, b]$.


