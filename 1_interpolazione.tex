\chapter{Interpolazione}
Avendo alcuni punti, si considerano i seguenti problemi:
\begin{itemize}
  \item Ricostruire una traiettoria passanti per i punti assegnati
  \item approssimare una funzione complessa nota in alcuni punti una più semplice come un polinomio
  \item calcolare il valore di un integrale definito di una funzione di cui non conosciamo facilmente ina primitiva
  ad esempio approssimandola con un polinomio
\end{itemize}

\section{Teorema}
Se $(x_i,y_i)$ con $i= 0, \dots, n$, sono $n + 1$ punti tali che $x_i \neq x_j$ se $i \neq j$, esiste
ed è unico il polinomio $p_n(x)$ di grado al più $n$ tale che:
\begin{equation}
  p_n(x_i)=y_i, \quad i = 0, \dots, n
\end{equation}

\section{Dimostrazione}
Consideriamo il generico polinomio di grado $n$:
\begin{equation}
  p_n(x) = a_0 + a_1x + \dots + a_nx^n
\end{equation}

ed imponiamo le $n + 1$ condizioni (vincoli)
\begin{equation}
  \begin{cases}
  a_0 + a_1x_0 + \dots + a_nx_0^n = y_0 \\
  a_0 + a_1x_1 + \dots + a_nx_1^n = y_1 \\

  \dotfill \\

  a_0 + a_1x_n + \dots + a_nx_n^n = y_n
  \end{cases}
  \label{eq:sistema_lineare_da_risolvere}
\end{equation}

I parametri (coefficienti) incogniti $a_0, a_1, \dots, a_n$ sono soluzione del sistema lienare \ref{eq:sistema_lineare_da_risolvere} di ordine $n + 1$, in forma matriciale
si può scrivere come:

\begin{equation}
  \begin{array}{c}
    \begin{pmatrix}
      1 & x_0 & \cdots & x_0^n \\ 
      1 & x_1 & \cdots & x_1^n \\ 
      \dotfill & \dotfill & \dotfill \\
      1 & x_n & \cdots & x_n^n 
    \end{pmatrix}

    \begin{pmatrix}
      a_0 \\
      a_1 \\
      \dotfill \\
      a_n
    \end{pmatrix}

    =

    \begin{pmatrix}
      y_0 \\
      y_1 \\
      \dotfill \\
      y_n
    \end{pmatrix}

    ,

    \quad Va = y


  \end{array}
\end{equation}

La matrice $V$ è detta \textbf{matrice di Vandermonde}, essa risulta \textbf{non singolare} se e soltanto se 
\textbf{il vettore nullo $0$ è la sola soluzione} del sistema omogeneo.

\begin{equation}
  Va = 0
\end{equation}

Quindi se il vettore $a = [a_0, a_!, \dots, a_n] \neq 0$ possiamo costruire il polinomio di grado $n$:
\begin{equation}
 p_n(x) = a_0 + a_1x + \dots + a_nx^n
\end{equation}

Questo polinomio soddisfa anche la condizione:
\begin{equation}
 p_n(x_i) = 0, \quad i = 0, 1, \dots, n
\end{equation}

Cioè il polinomio $p_n(x)$ di grado $n$ avrebbe $n + 1$ zeri (asurdo, vedi teorema fondamentae dell'algebra), di conseguenza
\textbf{il vettore $a$ deve essere il vettore nullo e quindi la matrice $V$ è non singolare.}

In alternativa si può verificare che la matrice $V$ è non singolare mediante il determinante, che se non nullo, il 
sistema ha una e una sola soluzione, quindi il polinomio $p_n(x)$ esiste ed è unico.

\section{Costruzione polinomio interpoaltore}

Per ottenere una base che riduca il numero di calcoli cerco una matrice $V \equiv I$.

Parto con il costruire i polinomi Lagranciani:
\begin{equation}
  \begin{array}{c}

    L_0(x_i) = \delta_{0, i}

    =

    \begin{cases}
      1 \quad \text{se} \quad i = 0 \\ 
      0 \quad \text{se} \quad i \neq 0
    \end{cases}

    i = 0, 1, 2

  \end{array}
\end{equation}

\begin{equation}
  \begin{array}{c}

    L_1(x_i) = \delta_{1, i}

    =

    \begin{cases}
      1 \quad \text{se} \quad i = 1 \\ 
      0 \quad \text{se} \quad i \neq 1
    \end{cases}

    i = 0, 1, 2

  \end{array}
\end{equation}

\begin{equation}
  \begin{array}{c}

    L_2(x_i) = \delta_{2, i}

    =

    \begin{cases}
      1 \quad \text{se} \quad i = 2 \\ 
      0 \quad \text{se} \quad i \neq 2
    \end{cases}

    i = 0, 1, 2

  \end{array}
\end{equation}


$L_0(x)$ si deve annullare in $x_1$ e $x_2$, quindi:
\begin{equation}
 L_0(x) = a_0(x - x_1)(x - x_2)
\end{equation}

Con $a_0$ costante arbitraria non nulla e imponendo che $L_0(x_0) = 1$ ottengo:
\begin{equation}
  L_0(x_0) = a_0(x - x_1)(x - x_2) = 1 \quad \rightarrow \quad a_0 = \displaystyle\frac{1}{(x - x_1)(x - x_2) }
\end{equation}

Ripetende lo stesso procedimento per $L_1(x)$ e $L_2(x)$ ottengo:
\begin{equation}
 L_0(x) = \displaystyle\frac{(x - x_1)(x - x_2)}{(x_0 - x_1)(x_0 - x_2)}, \quad L_0(x_0) = 1, L_0(x_1) = 0, L_0(x_2) = 0
\end{equation}

\begin{equation}
 L_1(x) = \displaystyle\frac{(x - x_0)(x - x_2)}{(x_1 - x_0)(x_1 - x_2)}, \quad L_1(x_0) = 0, L_1(x_1) = 1, L_1(x_2) = 0
\end{equation}

\begin{equation}
 L_2(x) = \displaystyle\frac{(x - x_0)(x - x_1)}{(x_2 - x_0)(x_2 - x_1)}, \quad L_2(x_0) = 0, L_2(x_1) = 0, L_2(x_2) = 1
\end{equation}


Ciascun polinomio è di secondo grado, inoltre i polinomi $L_0(x)$, $L_1(x)$ e $L_1(x)$ sono lineramente indipendenti.

Quindi posso esprimere un generico polinomio di secondo grado come combinazione dei tre nuovi polinomi costruiti:
\begin{equation}
 p_2(x) = a_0L_0(x) + a_1L_1(x) + a_2L_2(x) \quad a_0, a_1, a_2 \in R
\end{equation}

Assegnati tre valori di $y$ possiamo costruire il polinomio interpolatore $p_2(x)$ usando la nuova base Langrangiana:
\begin{equation}
  \begin{cases}
        a_0L_0(x_0) + a_1L_1(x_0) + a_2L_2(x_0) = y_0 \\ 
        a_0L_0(x_1) + a_1L_1(x_1) + a_2L_2(x_1) = y_1\\ 
        a_0L_0(x_2) + a_1L_1(x_2) + a_2L_2(x_2) = y_2
  \end{cases}
\end{equation}

Mediante il polinomio interpolatore posso scrivere i punti $(x_0, y_0)$, $(x_1, y_1)$ e $(x_2, y_2)$ come:

\begin{equation}
  p_2(x) = y_0L_0(x) + y_1L_1(x) + y_2L_2(x)
\end{equation}
