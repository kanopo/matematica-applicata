\chapter{Polinomio interpolatore di Lagrange}

Il polinomio interpolatore di Lagrange si presenta nella forma:
\begin{align}
  L_i = \prod_{k=0, k\neq i}^n \frac{x - x_k}{x_i - x_k}
\end{align}

Invece il polinomio interpolatore costruito grazie ai polnomi di Lagrange è:

\begin{align}
  p_n(x) = \sum_{i=0}^n y_i L_i(x)
\end{align}

La caratteristica del polinomio di Lagrange è che:
\begin{align}
  L_i(x_j) = \begin{cases}
    1 & \text{se } i = j \\
    0 & \text{se } i \neq j
  \end{cases}
\end{align}

Il polinomio interpolatore di Lagrange si può riscrivere come:
\begin{align}
  p_n(x)  &= \sum_{i=0}^n y_i L_i(x) \\
          &= \omega_n(x) \sum_{i = 0}^n \frac{\beta_i}{x-x_i}
\end{align}

Dove $\omega_n(x)$ è:
\begin{align}
  \omega_n(x) = \prod_{i=0}^n (x-x_i)
\end{align}
