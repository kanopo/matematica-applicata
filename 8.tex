\chapter{Non lineare}
Uno dei problemi pi\`u comuni in matematica \`e quello di trovare le radici di un'equazione.

\section{Bisezione}

\subsection{Teorema degli zeri per funzioni continue}
Sia $f : [a, b] \rightarrow R$, continua in $[a, b]$ e $f(a)f(b) < 0$. Allora esiste $x \in [a, b]$ tale che $f(x) = 0$.



Iterando il controllo $f(a)f(b) < 0$ e dividendo l'intervallo a met\`a, si pu\`o trovare una radice di $f$.

$c = \frac{a+b}{2}$ \`e il punto medio dell'intervallo $[a, b]$.

Se $f(a)f(c) < 0$ allora la radice \`e nell'intervallo $[a, c]$ ed itero la procedura su questo intervallo.




\subsection{Tolleranza}
Fissando una tolleranza $\epsilon$, la procedura si interrompe quando:
\begin{align}
  | \alpha - x_k | \leq \epsilon
\end{align}

\begin{align}
  k \geq \frac{\log_{10} \frac{b-a}{\epsilon}}{\log_{10} 2} \approx 3.32 \log_{10} \frac{b-a}{\epsilon}
\end{align}

Si capisce che in media sono necessarie 3.3 bisezioni per migliorare di una cifra significativa la precisione della radice.


\newpage
\section{Secanti}

Costrusco una successione $\{x_k\}$ tale che $\forall k \geq 1$, il punto $x_{k+1}$ \`e lo \textbf{zero} della
retta \textbf{secante} che passa per i punti:
\begin{center}
  \begin{tabular}{| c | c |}
  \hline
    $x$ & $f(x)$ \\
  \hline
    $x_{k-1}$ & $f(x_{k-1})$ \\
    $x_k$ & $f(x_k)$ \\
  \hline
  \end{tabular}
\end{center}


\begin{figure}[h!]
  \centering
  \includegraphics[width=0.4\textwidth]{images/secandi.png}
\end{figure}


Se la funzione \`e \textbf{convessa} o \textbf{concava} $\in [a, b]$, la successione converge a $\alpha$ in modo 
monotono (crescente o decrescente).

\section{Corde}
Il metodo delle corde \`e il caso generale alla base del metodo delle secanti, la $x$ successiva si calcola:
\begin{align}
  x_{k+1} = x_k - \frac{b-a}{f(b)-f(a)} f(x_k), \quad k \geq 0
\end{align}


\section{Tangenti}
Si pu\`o applicare il metodo delle tangenti solo se la funzione \`e \textbf{derivabile} in $[a, b]$.
Questo metodo \`e molto veloce a convergere, ma \`e molto sensibile alla scelta del punto di partenza $x_0$.


$x_0$ dato:
\begin{align}
  x_{k+1} = x_k - \frac{f(x_k)}{f'(x_k)}, \quad k \geq 0
\end{align}


\begin{figure}[h!]
  \centering
  \includegraphics[width=0.4\textwidth]{images/tangenti.png}
\end{figure}
